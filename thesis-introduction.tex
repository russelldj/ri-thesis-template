% -*- mode:LaTex; mode:visual-line; mode:flyspell; fill-column:75-*-

\chapter{Introduction} \label{secIntro}
\section{Motivation}
Drones are becoming increasingly useful in a variety of monitoring situations, from agriculture and forestry, to geology, to wildlife monitoring. In many contexts, the area of interest cannot be surveyed exhaustively due to constraints on imposed by flight speed and battery limitations. Therefore, many methods choose a subset of locations to visit which still allow the system to model the environment well, under a given metric. This problem goes by many names, including informative path planning (IPP), adaptive sampling, ergodic search, etc.

\section{}
%
%\section{Metric uncertainty}
%Despite the thorough treatment of uncertainty in the robotics community \cite{Thrun2002ProbabilisticRobotics}, it is commonly accepted that fundamental modeling assumptions are frequently violated \cite{SLAM course lectures}. This pattern is also repeated in the computer vision community, where uncertainty models are poorly calibrated \cite{something}. Often, these limitations are often ignored, because solutions work well enough in practice and it allows us to leverage powerful mathematical tools. This work argues that many situations require only 
%
%\section{Searching for high-value quantities }
%
%\section{How to use this template}
%
%\subsection{Algorithms}
%
%\input{algorithms/alg-template.tex}
%
%\section{Proposed experiments}
%\subsection{Estimating bias maps}