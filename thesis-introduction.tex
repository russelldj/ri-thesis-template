% -*- mode:LaTex; mode:visual-line; mode:flyspell; fill-column:75-*-

\chapter{Introduction} \label{chapIntroduction}
\section{The Importance of Forests}

Forests impact many aspects of our life on earth, ranging from the composition of the atmosphere, purification of water, moderating local temperature, and contributing to fire risk \cite{IPCC2019ClimateReport}. Unfortunately, they are under threat from a variety of factors including climate change, invasive species, fire, and direct human pressures. This is causing forests to change at an unprecedented rate. In light of these rapid changes, it is critical that we have up-to-date information to inform decisions such as habitat preservation, sustainable timber operations, forest fire mitigation, and carbon sequestration. In this work we specifically study aspects of forest fire mitation and carbon sequestration, though the approaches aim to be generic enough to scale to other applications. 

\subsection{Forest Fire Mitigation} 
Destructive forest fires have increased dramatically over the past decades \cite{spreading_like_wildfire, ayanz2021, nfn2022}. This is due in large part to climate change, which leads to hotter and drier weather along with stronger winds \cite{spreading_like_wildfire}. This has also led to increased forest mortality from pests expanding their range, such as the mountain pine beetle in the Western US \cite{Jenkins2014AndFuels}. Humans have also contributed more directly to fires by suppressing small fires which causes fuel to build up over time. Finally, there is an increase in ignition sources from careless human activity and infrastructure such as power lines. These fires are now more dangerous to humans and property because of their scale and speed as well as increasing habitation in close proximity to forests, termed the urban-woodland interface. The ecological consequences of fire are also increasingly dire. Historical fires were a natural part of the ecosystem and the vegetation was able to regenerate due to surviving trees and un-burned seeds. The intensity of modern fires completely destroys all vegetation, making it much harder to for regions to regrow. This can lead to erosion and eventual transition from forest to grassland.

It is becoming increasingly clear that reactive firefighting is insufficient to combat fires of this magnitude and preemptive mitigation efforts are required. One way to actively reduce the risk of destructive fires is by fuel management, or removing dense understory vegetation \cite{Fire2021FuelsManagement, WildlandFireResiliencyProgram20214Plan, Agriculture2019HazardousComplex}. This is a challenging problem due to the sheer area of forested land and the limited resources currently put toward preemptive efforts \cite{spreading_like_wildfire}.
There has been increasing government interest in technological innovation, with the USDA \cite{USDA2023USDAGrant} and NASA \cite{SPSO2023Research2023} releasing grant funding opportunities. These specifically address the need for pre-fire understanding and mitigation efforts. 
Since feul management is physically demanding and requires specilized knowledge, there are increasing concerns about labor shortages \cite{CommisionGlobalDivision}. This has led proposed robotic systems that can supplement human efforts by semi-autonomously removing vegetation \cite{couceiro2019semfire}. In many of these applications, a key step is understanding the current location of fuel in the environment. This can be useful for prioritizing which regions to target, especially when using an automated system that may have limited situational awareness. Furthermore, it can inform fire-fighting strategies if the region succumbs to fire. Therefore, one of our main objectives is to map what type of vegitation is where in the environment.

\subsection{Forest Carbon Sequestration Prediction}
The other problem we explore in this work relates to assessing carbon sequestration in forests. This is relevant because forest biomass is a key carbon sink \cite{Griscom2017NaturalSolutions}. Understanding this sequestration is important for our general understanding of likely climate futures, and also to aid in the fight against climate change. Carbon offsets or "credits" have emerged as a contentious tool in the fight against climate change and the eventual goal of net-zero emissions. They allow an entity (an individual, corporation, or government) to pay a carbon credit vendor to offset their emissions by sequestering or stopping the generation of a comparable amount of emissions. This is motivated by the unfortunate reality that change cannot happen immediately and some activities are much more challenging to de-carbonize than others. It is expected that demand for carbon credits will rise by a factor of 50 by 2050 according to a McKinsey report \cite{Blaufelder2021AChallenge}. 

Unfortunately there is widespread skepticism that this process truly offset the released carbon due to a variety of factors. The first is "carbon leakage" or the fact that carbon may be temporarily sequestered but released over time. This is especially relavebt for nature-based carbon capture solutions, such as when a forest burns due to wildfire. The second is "additiveity", which states that a carbon credit may be issued even though a behavior wasn't changed. For example, a credit could be issued for preserving the carbon in a forest, even though the carbon would not have been released even if the credit were not issued. The final issue is simple miss-estimation in the amount of carbon sequestered. Unfortunately, it has been shown that this error is a systematic overestimate of the true carbon stock \cite{Badgley2022SystematicProgram,West2020OverstatedAmazon}. While technology cannot address all the issues with carbon accounting, it can make validation more scalable and objective. In this work, we take a first step toward automated carbon estimation by accurately detecting individual trees in satellite imagery. This can allow us to apply tree-level modeling based on predicted attributes such as species and height to estimate biomass.    


\section{Current Approaches for Informing Forest Management}
\subsection{Manual Forestry Measurements}
Understanding forests is still a largely manual process where foresters go into the field and measure various quantities such as tree height, diameter, density, species. In commercial contexts this is often called timber cruises \cite{ServiceFSHHANDBOOK} and in ecology it is often called forest inventories \cite{USForestServiceDepartmentofAgriculture2016FORESTPLOTS}. This process is a extremely laborious and means that only a limited area can be surveyed.

\subsection{Drone Surveys}
Commodity drone data refers to data which can be acquired with only commercially-available tools and no specialized domain knowledge. In practice, a valuable and easy-to-collect source of data is GPS-tagged images collected in a pre-defined exhaustive survey. This can be obtained by even some of the smallest drones, such as DJI Mavic, which costs approximately \$1,600\footnote{\href{https://store.dji.com/product/dji-mavic-3-classic}{DJI Mavic}}. Commercial flight planners are a tool that allows a user to quickly define a survey region and fly over it in a geometric pattern. A common option is "lawnmower coverage" where a drone flies in a straight line, shifts over, and flies back the opposite direction. 


\subsection{Remote Sensing}
Remote sensing data is captured by sensors onboard satellites or airplanes and is rapidly becoming a critical tool for environmental monitoring \cite{Parra2022RemoteMonitoring}. This is because of the large spatial regions that are observed by these sensors which can span up to global converge in the case of some satellites. This data can take many forms ranging from light detection and ranging (LiDAR) \cite{LiDARForestryBeland2019}, synthetic aperture radar (SAR) \cite{Hall2020WhatEarthdata}, and electro-optical (EO) data.


%Most traditional cameras only capture three different bands of light, red, green and blue. In contrast, many space-borne sensors capture more spectral bands and are termed either multi- or hyper-spectral. Multispectral data has coarse bands with gaps between them, while hyperspectral data has many small bands which observe a near-continuous spectral signal \cite{Lu2019ComparingProperties}. These sensors vary widely in spatial resolution with legacy government satellites such as Landsat 8 and Sentinel 2 having resolutions of multiple meters per pixel. More recent commercial offerings have pushed the resolution to the sub-meter range. 


%In this work, we focus on optical data, since it is conceptually most similar to what we capture from drones. This data is collected by an electro-optical sensor which takes images of the earth. Then, the images are registered together and referenced into absolute geospatial coordinates. This process relies on the estimated pose of the platform when the image was taken, the height of the ground, and manual corrections. From there, an orthographic render is generated. This is a projection of the data into a top-down view, which is commonly aligned with the axes of geospatial coordinate system used to define the location. Many satellite data sources capture bands outside of the visible wavelength. Sensors which capture a x to y number of bands are termed multi-spectral and sensors which captured y to z bands are termed hyper-spectral. 
%\begin{itemize}
%    \item talk about what what sat is 
%    \item Sat data has a broad extent and may be too low res
%    \item Reannalysis has shown that accuracy is often very low \cite{} 
%    \item Data products from crewed aircraft has higher resolution but require substaintial investment and planning
%\end{itemize}

%Another option is "grid coverage" where a lawnmower pattern is executed twice, in perpendicular directions. An important consideration when executing these surveys is the drone altitude, which represents a trade-off between coverage ability and spatial resolution. While drones can in theory fly to a substantial altitude, they are restricted in the US to 400ft and below due to concerns about interfering with crewed aircraft. The final consideration is front- and side-overlap. Front overlap refers to the fraction of the image which is observed in two consecutive frames captured as the drone is flying forward. The side overlap refers to how much overlap there is between neighboring flight lines.
