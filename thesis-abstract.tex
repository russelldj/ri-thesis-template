% -*- mode:LaTex; mode:visual-line; mode:flyspell; fill-column:75-*-

% Special indentation for abstract.
\setlength{\parskip}{1em}
\setlength{\parindent}{0em}

\noindent
There are many robotic tasks which involve information gathering in an unknown or partially-known environment. It is often possible to beat a naive exaustive exploration by using \textit{informative path planning} (IPP) which leverages observations online to developed a plan to observe the samples which are predicted to be most useful. In the simplest setting, the most useful sample is often the one which yeilds the most information about the environment. In some settings, the search may be targeted, and some portion of the environment deserves more attention.

This thesis specifically focuses on the problem of a drone exploring an unknown region for a specific scientific mission. As motivating examples this thesis examines real data for three classification problems: desert minerology, benthic cover classification, and forest fire fuel identification. Our goal is to develop a straightforward approach which can be deployed on a consumer grade drone or other robotic agent and requires no online trajectory replanning. This enables end-users (geologists, oceonographers, foresters, etc.) to use the system without extensive system integration and requisite robotics knowledge. 

Our main conceptual insight is that it's possible to obtain overhead low-resolution imagery of each of these scenes. A key technical insight is that without any other information, our goal is to sample a diverse set of points both in the spatial and spectral sense. Therefore, we cluster our data based on spatial and spectral features and extract one candidate location per cluster. Then we run a global optimization to determine which subset of points are valuable to visit. Finally, we run a traveling salesman optimization to determine an efficient path which visits all the chosen locations. This plan can be used on any commodity research platform such as a UAV or AUV. After completing the mission, the new information can optionally be used to generate a new plan.