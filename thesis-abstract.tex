% -*- mode:LaTex; mode:visual-line; mode:flyspell; fill-column:75-*-

% Special indentation for abstract.
\setlength{\parskip}{1em}
\setlength{\parindent}{0em}
Drones and remote sensing can provide observations of forests at scale, but this raw data needs to be interpreted to further our scientific understanding and inform effective management decisions. This thesis studies two problems under the realistic constraint of limited domain-specific training data: tree detection for understanding carbon sequestration and vegetation mapping for forest fire mitigation.

For tree detection, we process the drone data using structure from motion and then register it to remote sensing imagery. Then, we compare different strategies for using a deep learning detector with these modalities and limited training data.
For vegetation mapping, we show that we can localize fuel that causes forest fires using image-based semantic segmentation trained on very few examples and LiDAR-based geometric reasoning. Finally, we introduce RAPTORS, a novel algorithm that plans where to collect sparse drone observations based on existing remote sensing data. We show that training a remote sensing-based vegetation classification model on observations from RAPTORS is more effective at identifying rare classes than training on observations from a coverage-based approach. 
Overall, these experiments show how using machine learning, data harmonization across scales, and intelligent sampling can facilitate automated forest understanding with limited training data.
%These approaches could inform data-driven forest management, which is critical in this time of rapid environmental change.

%We explore how to best detect trees using co-registered data from remote sensing, drone surveys, and manual observations of tree locations. We use a pre-trained deep learning model to detect trees in drone and aerial data and propose a multi-scale fine tuning process to increase performance. 
%We first fine-tune the model for the drone data, using the sparse but accurate field reference measurements, then use the predictions on the entire region surveyed by the drone to fine-tune a model for remote sensing data that has broad coverage.
%To map vegetation structure and type, we combine image-based semantic segmentation models and geometric reasoning from SLAM to build metric-semantic maps of the environment. 
%Finally, we introduce RAPTORS, a novel informative path algorithm that that is suitable for low-cost consumer drones because it can plan the entire mission before takeoff.
%RAPTORS leverages existing remote sensing data using unsupervised feature extraction, determines uncertain regions with Gaussian Process modeling, and plans a path using a sampling-based approach.

%We find that tree detections on remote sensing data are improved by the multi-stage fine-tuning process compared to the base model or single-stage training .
%We show that the mapping system can generate a map that accurately classifies vegetation type using a surprisingly-few number of training images.
%We evaluate RAPTORS by planning drone flights to collect observations of land cover, to train a classification model using remote sensing data. We demonstrate that a model trained on observations from RAPTORS generates better predictions of rare classes than a model trained on observations from a baseline coverage planner. A commonality of these experiments is that a small quantity of highly-relevant data can be critical for automating forest understanding. Additionally, combining observations from different modalities can yield more accurate and scalable predictions, which is critical in this time of increasing environmental change.





%Forests are undergoing rapid changes driven climate change, land use changes, and invasive species. Humans are often tasked with management decisions, with goals such as preserving biodiversity, improving carbon sequestration, or reducing the risk of wildfire. Unfortunately, land managers rarely have high quality quantitative data to inform these decisions. Examples include the location or species of tree or a map of the predominant vegitation type.
%Field work is labor intensive and only allows a small subset of a region to be surveyed. Satellite data can provide effectively-unlimited coverage, but suffers from low spatial resolution which impedes the quality of automated prediction systems. Unmanned aerial systems or "drones" represent an effective middle ground between this two currently-used approaches. Drones can cover a larger area than a forester can with minimal human effort and provide orders-of-magnitudes higher resolution than satellite data. Furthermore, drones can be deployed on demand, to provide timely information about a recent event such as a fire, as opposed to satellite data which can take weeks or months to re-observe a region. 

%The robotics community has sought to apply drones to forestry in a variety of settings. However, this work focuses on highly-engineered multi-sensor drones that perform extensive online processing. This technology is simply not ready for deployment by land managers, who frequently lack dedicated technical staff. However, the commercial drone market has made great strides in usability over the last decade and drone surveys are pervasive in many domains such as agriculture, surveying, and inspection. Due to the robust commercial and open source ecosystem, forestry is seeing an increasing adoption of automated drone surveys. These are often executed autonomously using a coverage plan which is configured before the mission. Then, data is fed into a photogrametery software which produces colored 3D meshes or pointclouds of the scene, using geometric computer vision. 

%This thesis aims to make two contributions. The first is a flexible full-mission planner which prioritizes observing a diverse set of observations. This is especially useful in situations where there is imbalanced distribution of classes or a classes are clumped spatially. The second is supplementing the existing geometric mapping with semantic information. This could be something like the type or species of the vegetation.

\noindent