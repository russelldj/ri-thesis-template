% -*- mode:LaTex; mode:visual-line; mode:flyspell; fill-column:75-*-

% Special indentation for abstract.
\setlength{\parskip}{1em}
\setlength{\parindent}{0em}

Forests are undergoing rapid changes driven climate change, land use changes, and invasive species. Humans are often tasked with management decisions, with goals such as preserving biodiversity, improving carbon sequestration, or reducing the risk of wildfire. Unfortunately, land managers rarely have high quality quantitative data to inform these decisions. Examples include the location or species of tree or a map of the predominant vegitation type.
Field work is labor intensive and only allows a small subset of a region to be surveyed. Satellite data can provide effectively-unlimited coverage, but suffers from low spatial resolution which impedes the quality of automated prediction systems. Unmanned aerial systems or "drones" represent an effective middle ground between this two currently-used approaches. Drones can cover a larger area than a forester can with minimal human effort and provide orders-of-magnitudes higher resolution than satellite data. Furthermore, drones can be deployed on demand, to provide timely information about a recent event such as a fire, as opposed to satellite data which can take weeks or months to re-observe a region. 

The robotics community has sought to apply drones to forestry in a variety of settings. However, this work focuses on highly-engineered multi-sensor drones that perform extensive online processing. This technology is simply not ready for deployment by land managers, who frequently lack dedicated technical staff. However, the commercial drone market has made great strides in usability over the last decade and drone surveys are pervasive in many domains such as agriculture, surveying, and inspection. Due to the robust commercial and open source ecosystem, forestry is seeing an increasing adoption of automated drone surveys. These are often executed autonomously using a coverage plan which is configured before the mission. Then, data is fed into a photogrametery software which produces colored 3D meshes or pointclouds of the scene, using geometric computer vision. 

This thesis aims to make two contributions. The first is a flexible full-mission planner which prioritizes observing a diverse set of observations. This is especially useful in situations where there is imbalanced distribution of classes or a classes are clumped spatially. The second is supplementing the existing geometric mapping with semantic information. This could be something like the type or species of the vegetation.

\begin{itemize}
    \item TODO make this more about what I did
\end{itemize}

\noindent