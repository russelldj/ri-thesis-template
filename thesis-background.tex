% -*- mode:LaTex; mode:visual-line; mode:flyspell; fill-column:75-*-

\chapter{Background} \label{chapBackground}

\section{Different lenses on IPP}
IPP can be conceptualized using insights or frameworks from many different fields...
\begin{itemize}
    \item Ergodic search
    \item Adaptive sampling
    \item Active learning 
    \item Sensor placement
    \item Target search and security games
\end{itemize}

\section{Current methods}
\subsubsection{Online methods}
\begin{itemize}
    \item \textbf{Real robots} Alberto's Planetary \cite{Kodgule2019Non-myopicMeasurements}, Alberto's Coral \cite{Candela2021}, Yogi's Coral \cite{Jamieson2020ActiveEnvironments}, Marija's Multi-resolution \cite{Stache2021AdaptiveSegmentation}, Marija's framework one \cite{Popovic2020}
\end{itemize}


\subsubsection{Offline/batch methods}
\begin{itemize}
    \item \textbf{Ergodic} Ananya's sparse sensing \cite{RaoSparseOptimization}
    \item \textbf{Generic} Sankalp's \cite{Arora2017RandomizedConstraints}, Daniela Rus' Correlated orienteering \cite{Yu2016CorrelatedTasks}
    \item \textbf{TIGRIS} \cite{Moon2022}
    In this work the authors use a sampling based approach. The chance of taking a sample is biased toward those locations which have higher information content. Importantly, this work takes into account the s
\end{itemize}

\section{The gap which needs to be filled}
\begin{itemize}
    \item Batch planners don't take much contextual information such as satellite imagery 
    \item Online planners are very complex to implement
\end{itemize}

