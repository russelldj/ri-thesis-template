% -*- mode:LaTex; mode:visual-line; mode:flyspell; fill-column:75-*-

\chapter{Conclusions} \label{chapConclusions}
\section{Key Takeaways}
\begin{itemize}
    \item Structure from motion is a powerful tool to build 3D maps from drone images with minimal assumptions
    \item A small amount of annotated data can be very useful to train deep learning models for a given natural environment
    \item Predicting trees at increasingly-low resolutions yields better results 
\end{itemize}
\section{Contributions}
\begin{itemize}
    \item A study of tree detection in data at multiple scales 
    \item A novel long-horizon informative path planner for drone surveys
\end{itemize}
\section{Future work}
\begin{itemize}
    \item Adding semantics to meshes 
    \item Studying tree detection with more rigorous ground truth data
    \item Adding height information to semantics and tree detection images
    \item Explore different approaches to dimensionality reduction, such as VAE
    \item Automatically registering data at different scales using trees as features
    \item Evaluating the performance of different feature extractors
\end{itemize}

Experiments on this topic are ongoing. We propose to add two more experiments to this list. These involve fine-tuning a model for aerial data on predicted tree locations from the drone data. In one case we used predictions from the pretrained model and in the other case we use predictions from the fine-tuned model. We hypothesis that this multi-stage approach will increase performance because of the increased diversity of training examples.  

Quantitative evaluation is also ongoing. To get an realistic assessment of the quality, we need a test set that was not within the drone data used in any of the experiments. Unfortunately, we need the drone data to label the ground truth tree locations. Therefore, we plan conduct this experiment with a two-fold evaluation scheme using the Stowe and Weisner datasets that were collected in the same region of Vermont US. This will require registering 
In each case, we generate predictions on one location, and if applicable, use the other location for training. We evaluate the performance based on precision and accuracy, with a given IoU threshold used to determine whether a prediction matches a ground truth. We also evaluate the average IoU between predictions and groundtruth.


